\documentclass{article}

\usepackage[serbian]{babel}
\usepackage[a4paper,top=2cm,bottom=2cm,left=3cm,right=3cm]{geometry}

\usepackage{graphicx}
\usepackage{svg}
\usepackage{float}
\graphicspath{ {./images/} }

\usepackage{titlesec}
\usepackage{amsmath}
\usepackage{parskip}
\usepackage[all]{nowidow}
\usepackage[
  backend=bibtex,
  style=authoryear,
  citestyle=authoryear,
  maxcitenames=1,
  uniquename=false,
  hyperref=true
]{biblatex}

\addbibresource{references.bib}

\renewcommand{\nameyeardelim}{\addspace}

% Remove the period/dot after year in citations
% The period comes from Serbian locale's date format - override it
\DefineBibliographyExtras{serbian}{%
  \protected\def\mkbibdatelong#1#2#3{%
    \iffieldundef{#3}{}{\stripzeros{\thefield{#3}}%
      \iffieldundef{#2}{}{.~}}%
    \iffieldundef{#2}{}{\mkbibmonth{\thefield{#2}}%
      \iffieldundef{#1}{}{\addspace}}%
    \iffieldbibstring{#1}{\bibstring{\thefield{#1}}}{\stripzeros{\thefield{#1}}}}%
}

% Override labeldateparts format to not add period after year
\DeclareFieldFormat{labelyear}{#1}
\DeclareFieldFormat{extrayear}{}
\DeclareFieldFormat[misc,online,article,book,inbook,incollection]{labelyear}{#1}

% Redefine parencite year format
\renewbibmacro*{cite:labeldate+extradate}{%
  \iffieldundef{labelyear}
    {}
    {\printtext[bibhyperref]{%
       \printfield{labelyear}%
       \printfield{extrayear}}}}

\usepackage{tikz}
\usetikzlibrary{
    positioning,
    arrows.meta,
    fit,
    backgrounds
}
\setlength{\bibitemsep}{0.6em}
\usepackage[colorlinks=true, allcolors=blue]{hyperref}
\usepackage{bookmark}
\usepackage{csquotes}
\usepackage[most]{tcolorbox}
\newtcolorbox{promptbox}{
  enhanced,
  colback=gray!5,
  colframe=black!60,
  boxrule=0.5pt,
  arc=3pt,
  left=6pt,
  right=6pt,
  top=6pt,
  bottom=6pt,
  breakable,
  width=0.9\textwidth,
  center
}

\makeatletter
\newcommand{\thetitle}{\@title}
\makeatother

\newcommand{\university}{Univerzitet u Banjoj Luci}
\newcommand{\faculty}{Prirodno-matematički fakultet}
\newcommand{\department}{Studijski program Matematika i informatika}

\newcommand{\subject}{Uvod u vještačku inteligenciju}
\newcommand{\worktype}{Seminarski rad}

\newcommand{\student}{Vladimir Mijić}
\newcommand{\mentor}{prof. dr Marko Đukanović}

\title{Analiza i poređenje metaheurističkih i egzaktnih pristupa za rješavanje problema bojenja particija}

\begin{document}

\begin{titlepage}
  \begin{center}
    \includegraphics[width=0.25\textwidth]{pmf_logo.png}\\[2ex]
    
    {\Large \university}\\[1ex]
    {\large \faculty}\\[1ex]
    {\normalsize \department}\\[3ex]
    
    {\large \subject}\\[1ex]
    {\large \textbf{\worktype}}\\[6ex]
    
    \vfill
    {\huge \bfseries \thetitle}
    \vfill
    
    \begin{minipage}[t]{0.48\textwidth}
      \raggedright
      \textbf{Student:}\\
      \student
    \end{minipage}%%
    \hfill%%
    \begin{minipage}[t]{0.48\textwidth}
      \raggedleft
      \textbf{Mentor:}\\
      \mentor
    \end{minipage}
    
    \vfill
    {\large Banja Luka, \today}
  \end{center}
\end{titlepage}

\tableofcontents
\newpage

\section{Uvod}

Problemi bojenja grafova decenijama se nalaze u samom fokusu istraživanja unutar teorije grafova i kombinatorne optimizacije. U svom standardnom obliku, \textit{problem bojenja čvorova grafa} (engl. \textit{Vertex Coloring Problem} -- VCP, ili \textit{Graph Coloring Problem} -- GCP) sastoji se u dodjeljivanju boje svakom čvoru grafa $G=(V,E)$ uz ograničenje da nijedna dva susjedna čvora ne mogu biti iste boje. Cilj je minimizovati ukupan broj upotrijebljenih boja, što je poznato kao hromatski broj grafa $\chi(G)$.

\textit{Problem bojenja particija} (engl. \textit{Partition Coloring Problem} -- PCP) predstavlja prirodnu, ali značajno složeniju generalizaciju VCP-a. Kod PCP-a, skup čvorova grafa je podijeljen na disjunktne grupe, tzv. particije. Za razliku od klasičnog bojenja, gdje se moraju obojiti svi čvorovi grafa, ovdje je zadatak izabrati tačno jedan čvor iz svake particije. Indukovani podgraf nad skupom izabranih čvorova zatim se boji tako da susjedni čvorovi nemaju istu boju, pri čemu je cilj minimizovati ukupan broj korištenih boja. U literaturi se PCP često naziva i \textit{selektivno bojenje grafa} (engl. \textit{Selective Graph Coloring}), čime se naglašava obavezna faza selekcije prije samog bojenja.

Ovaj problem je prvi put formalno definisan u radu \cite{liSimha2000pcp}, a primarnu motivaciju pronašao je u dizajnu optičkih mreža, konkretno u problemu rutiranja i dodjele talasnih dužina (engl. \textit{Wavelength-Division Multiplexing} -- WDM). U tom kontekstu, svaka particija predstavlja skup alternativnih puteva (ruta) za prenos signala, dok bojenje modeluje dodjelu frekvencija na način koji izbjegava međusobne smetnje. Pored telekomunikacija, PCP je relevantan i za širok spektar problema raspodjele resursa, pravljenja rasporeda (\textit{scheduling}), pozicioniranja antena i drugih srodnih optimizacionih zadataka.

S algoritamskog aspekta, problem bojenja particija predstavlja izazovan zadatak koji pripada klasi NP-teških problema. Kako bi se ova složenost bolje razumjela, prije formalnog definisanja problema neophodno je istaći razloge zbog kojih PCP prevazilazi složenost klasičnog problema bojenja grafova. U okviru PCP-a, izbor čvorova iz particija direktno određuje topologiju konflikata u grafu koji se boji. Neadekvatan izbor može rezultovati gustim podgrafom koji zahtijeva veći broj boja, dok pažljivo odabrana selekcija može značajno redukovati hromatski broj indukovanog podgrafa. Zbog ove neraskidive zavisnosti između selekcije i bojenja, PCP se može posmatrati kao optimizacioni problem sa dva međusobno zavisna nivoa odlučivanja.

Cilj ovog seminarskog rada je da definiše formalni okvir PCP-a, pruži pregled najznačajnijih algoritamskih pristupa u literaturi, te eksperimentalno uporedi efikasnost egzaktnih metoda sa savremenim metaheuristikama.

\subsection{Formalna definicija problema}

Neka je $G=(V,E)$ jednostavan neusmjeren graf. Neka je $\mathcal{P}=\{V_1, V_2, \dots, V_p\}$ particija skupa čvorova $V$ na $p$ disjunktnih nepraznih podskupova, tako da važi $V_i \cap V_j = \emptyset$ za $i \neq j$ i $\bigcup_{i=1}^p V_i = V$.

\textbf{Definicija 1 (Dozvoljena selekcija).} Skup $S \subseteq V$ naziva se \textit{dozvoljena selekcija} ako sadrži tačno po jedan čvor iz svake particije skupa $\mathcal{P}$, odnosno:
\[
|S \cap V_i| = 1, \qquad \forall i \in \{1, \dots, p\}.
\]

Neka je $G[S]$ \textit{indukovani podgraf} grafa $G$ definisan skupom čvorova $S$. Indukovani podgraf $G[S]$ sadrži samo one čvorove koji su izabrani dozvoljenom selekcijom, kao i sve ivice koje su u originalnom grafu postojale između tih čvorova.

\textbf{Optimizaciona verzija PCP-a.} Cilj je pronaći dozvoljenu selekciju $S$ takvu da je hromatski broj indukovanog podgrafa minimalan:
\[
\min_{S} \; \chi(G[S]).
\]

Minimalna vrijednost $\chi(G[S])$ preko svih dozvoljenih selekcija naziva se \textit{particioni hromatski broj} i označava se sa $\chi_P(G,\mathcal{P})$.

\textbf{Računarska složenost i NP-težina.}
Formalno, računarska složenost PCP-a može se okarakterisati polazeći od njegove veze sa klasičnim problemom bojenja čvorova grafa. Klasični problem bojenja čvorova grafa je specijalni slučaj PCP-a u kojem je svaka particija sastavljena od samo jednog čvora, tj. $V_i = \{v_i\}$. U tom slučaju, izbor čvorova je fiksiran ($S = V$), pa se PCP svodi na GCP. S obzirom na to da je GCP dokazano NP-težak problem, direktno slijedi da je i PCP NP-težak.


\subsection{Ilustrativni primjer}

Radi ilustracije problema bojenja particija i uticaja selekcije čvorova na proces bojenja, na slici \ref{fig:pcp-primjer} prikazan je primjer PCP-a sa tri particije. Konkretno, prikazana je selekcija $\{a_1, b_2, c_1\}$ koja inducira podgraf obojiv sa dvije boje. Iako je polazni graf gusto povezan, izbor čvorova direktno utiče na hromatski broj indukovanog podgrafa, što ilustruje suštinsku zavisnost između faze selekcije i procesa bojenja u PCP-u. Drugačiji izbor čvorova iz istih particija može dovesti do indukovanog podgrafa koji zahtijeva veći broj boja.

\begin{figure}[H]
\centering
\begin{tikzpicture}[
  v/.style={circle, draw, thick, minimum size=8mm, inner sep=0pt, font=\footnotesize},
  sel1/.style={circle, draw, very thick, fill=blue!25, minimum size=8mm, inner sep=0pt, font=\footnotesize},
  sel2/.style={circle, draw, very thick, fill=red!25, minimum size=8mm, inner sep=0pt, font=\footnotesize},
  e/.style={thick}
]

% Particija V1
\node[sel1] (a1) at (0,2) {$a_1$};
\node[v]    (a2) at (0,0.5) {$a_2$};

% Particija V2
\node[v]    (b1) at (2.5,2) {$b_1$};
\node[sel2] (b2) at (2.5,0.5) {$b_2$};

% Particija V3
\node[sel1] (c1) at (5,2) {$c_1$};
\node[v]    (c2) at (5,0.5) {$c_2$};

% Ivice
\draw[e] (a1)--(b1);
\draw[e] (a1)--(b2);
\draw[e] (a2)--(b1);
\draw[e] (a2)--(b2);

\draw[e] (b1)--(c1);
\draw[e] (b1)--(c2);
\draw[e] (b2)--(c1);
\draw[e] (b2)--(c2);

\draw[e] (a1)--(c2);
\draw[e] (a2)--(c1);
\draw[e] (a2)--(c2);

% Okviri particija
\begin{scope}[on background layer]
  \node[draw, dashed, rounded corners, fit=(a1)(a2), inner sep=6pt, label=below:$V_1$] {};
  \node[draw, dashed, rounded corners, fit=(b1)(b2), inner sep=6pt, label=below:$V_2$] {};
  \node[draw, dashed, rounded corners, fit=(c1)(c2), inner sep=6pt, label=below:$V_3$] {};
\end{scope}

\end{tikzpicture}

\caption{Primjer problema bojenja particija (PCP). Iz svakog skupa $V_i$ bira se tačno jedan čvor; izabrani čvorovi su označeni bojom i formiraju indukovani podgraf.}
\label{fig:pcp-primjer}
\end{figure}


\subsection{Pregled postojeće literature}

U radu \textcite{liSimha2000pcp} autori su prilagodili klasične heurističke algoritme bojenja, poput \textit{Largest-First} i \textit{Smallest-Last}, za potrebe particionisanih grafova. U istom radu definisan je i početni skup testnih instanci zasnovanih na realnim mrežnim topologijama, uključujući i \textit{Nsfnet} (\textit{National Science Foundation Net}).

U domenu egzaktnih metoda, značajan doprinos dali su \textcite{frotaEtAl2010branchcut}, koji su razvili \textit{branch-and-cut} algoritam zasnovan na formulaciji predstavnika. Iako je ovaj pristup omogućio rješavanje instanci srednje veličine, pokazao je ograničenja pri rastu gustine konflikata. Alternativni egzaktni pristup, zasnovan na \textit{branch-and-price} tehnici i Dantzig--Wolfe dekompoziciji, predstavljen je u radu \textcite{hoshinoEtAl2011branchprice}.

Novija literatura, poput rada \textcite{FURINI2018170}, donosi unapređenje ILP (engl. \textit{Integer Linear Programming}) formulacije i efikasne metaheuristike zasnovane na adaptivnoj pretrazi. U ovom radu korišćene su i objedinjene tri klase instanci koje se danas često koriste za poređenje algoritama:
\begin{itemize}
    \item \textbf{Random:} slučajno generisani grafovi sa različitim gustinama konflikata.
    \item \textbf{Nsfnet:} instance izvedene iz realne topologije internet mreže Sjedinjenih Američkih Država.
    \item \textbf{Ring:} grafovi sa prstenastom strukturom, karakteristični za određene topologije optičkih mreža.
\end{itemize}

U eksperimentalnom dijelu ovog seminarskog rada koristiće se navedene klase instanci radi evaluacije implementiranih algoritama.



\section{Metodologija i algoritmi}
Ovo poglavlje organizujemo tako da jasno razdvojimo (i) egzaktne metode, koje služe kao referentna tačka na malim instancama i daju dokazivo optimalna rješenja, i (ii) heurističke/metaheurističke metode, koje ciljaju veće instance gdje je optimalnost računski preskupa.


\subsection{Egzaktna metoda: model cjelobrojnog linearnog programiranja}

Egzaktno rješavanje problema bojenja particija podrazumijeva pronalaženje dozvoljene selekcije čvorova i odgovarajućeg bojenja takvog da je ukupan broj korištenih boja minimalan. S obzirom na to da je PCP NP-težak problem, potpuna pretraga prostora rješenja postaje neizvodljiva već za instance umjerene veličine. Zbog toga se u literaturi PCP često modeluje kao problem \textit{cjelobrojnog linearnog programiranja} (engl. \textit{Integer Linear Programming} -- ILP), što omogućava optimalno rješavanje manjih instanci i služi kao referentna tačka za poređenje sa heurističkim i metaheurističkim pristupima.

U ovom radu koristi se ILP formulacija zasnovana na dodjeli boja (engl. \textit{assignment formulation}), koja predstavlja standardni i intuitivan pristup za rješavanje problema selektivnog bojenja \parencite{frotaEtAl2010branchcut,FURINI2018170}. Posebna prednost ove formulacije jeste u tome što selekcija čvorova i njihovo bojenje nisu razdvojeni u dvije faze, već su objedinjeni u jedinstvenu odluku.

\subsubsection{Matematička formulacija modela}

Neka je $K$ gornja granica broja boja. U najgorem slučaju može se uzeti $K = p$, gdje je $p$ broj particija, budući da se uvijek može dodijeliti različita boja svakom izabranom čvoru.

Model koristi sljedeće binarne varijable:
\begin{itemize}
    \item $y_{v,c} \in \{0,1\}$ -- ima vrijednost 1 ako je čvor $v \in V$ izabran i obojen bojom $c \in \{1,\dots,K\}$, a 0 inače.
    \item $w_c \in \{0,1\}$ -- ima vrijednost 1 ako je boja $c$ upotrijebljena za bojenje bar jednog čvora, a 0 inače.
\end{itemize}

U ovoj formulaciji, selekcija čvorova i dodjela boja objedinjeni su u jedinstvenu odluku putem varijabli $y_{v,c}$, čime se izbjegava uvođenje eksplicitnih selekcionih varijabli i dobija kompaktniji ILP model.

\paragraph{Funkcija cilja:}
Cilj je minimizovati ukupan broj upotrijebljenih boja:
\begin{equation}
\min \sum_{c=1}^{K} w_c.
\end{equation}

\paragraph{Ograničenja modela:}
Kako bi rješenje bilo dopustivo, moraju biti zadovoljeni sljedeći uslovi.

\begin{enumerate}
    \item \textbf{Jedinstvena selekcija po particiji:}  
    Iz svake particije $V_i \in \mathcal{P}$ mora biti izabran tačno jedan čvor i dodijeljena mu tačno jedna boja:
    \begin{equation}
    \sum_{v \in V_i} \sum_{c=1}^{K} y_{v,c} = 1,
    \qquad \forall V_i \in \mathcal{P}.
    \end{equation}

    \item \textbf{Izbjegavanje konflikata:}  
    Ako su čvorovi $u$ i $v$ povezani ivicom u grafu $G$, oni ne mogu biti obojeni istom bojom:
    \begin{equation}
    y_{u,c} + y_{v,c} \le 1,
    \qquad \forall uv \in E,\ \forall c \in \{1,\dots,K\}.
    \end{equation}
    Ovo ograničenje obezbjeđuje da nijedna dva susjedna izabrana čvora ne dijele istu boju.

    \item \textbf{Praćenje upotrebe boja:}  
    Boja $c$ se smatra upotrijebljenom ako je dodijeljena bar jednom čvoru:
    \begin{equation}
    y_{v,c} \le w_c,
    \qquad \forall v \in V,\ \forall c \in \{1,\dots,K\}.
    \end{equation}

    \item \textbf{Eliminacija simetričnih rješenja:}  
    Kako bi se izbjegla ekvivalentna rješenja koja se razlikuju samo permutacijom boja i smanjio prostor pretrage, uvodi se ograničenje koje nameće fiksni redoslijed korišćenja boja:
    \begin{equation}
    w_c \ge w_{c+1},
    \qquad \forall c \in \{1,\dots,K-1\}.
    \end{equation}
\end{enumerate}

\subsubsection{Softverska implementacija i solveri}

Postavljeni ILP model implementiran je u programskom jeziku Python korišćenjem biblioteke \textit{Google OR-Tools}\footnote{\url{https://developers.google.com/optimization}}, dok je za njegovo rješavanje korišćen egzaktni MIP solver \textit{SCIP} (\textit{Solving Constraint Integer Programs}) \footnote{\url{https://www.scipopt.org}}, koji je široko zastupljen u akademskim istraživanjima.

S obzirom na eksponencijalnu složenost problema, u eksperimentalnom dijelu uvedeno je vremensko ograničenje od 300 sekundi po instanci. U slučajevima kada solver ne pronađe optimalno rješenje u zadatom vremenu, bilježi se najbolje pronađeno dopustivo rješenje, kao i pripadajući \textit{MIP gap}, koji predstavlja relativno odstupanje između dobijenog rješenja i najbolje poznate donje granice.

\pagebreak
\section{Eksperimentalna evaluacija}
Eksperimentalna evaluacija biće sprovedena na \textit{random}, \textit{nsfnet} i \textit{ring} instancama.

\pagebreak
\section{Diskusija}
% todo: dodati na kraju

\section{Zaključak}
% todo: dodati na kraju

\clearpage
\phantomsection
\printbibliography[heading=bibintoc, title={Literatura}]

\end{document}
